% The master copy of this demo dissertation is held on my filespace
% on the cl file serve (/homes/mr/teaching/demodissert/)

% Last updated by PMY on 3 March 2012

\documentclass[12pt,twoside,notitlepage]{report}

\usepackage{a4}
\usepackage{verbatim}
\usepackage{epsf}
\usepackage{sectsty}

\usepackage[xetex]{graphicx}
\usepackage{fontspec,xunicode}
\defaultfontfeatures{Mapping=tex-text,Scale=MatchLowercase}% , Numbers=OldStyle}
\setmainfont[Scale=1]{Sabon MT Std}% {Linux Libertine O}
\setsansfont{Myriad Pro Light}
\setmonofont{Monaco}
\allsectionsfont{\sffamily}

%%% ====================================================================
%%%   This file is freely redistributable and placed into the
%%%   public domain by Tomas Rokicki.
%%%  @TeX-file{
%%%     author          = "Tom Rokicki",
%%%     version         = "2.7k",
%%%     date            = "19 July 1997",
%%%     time            = "10:00:05 MDT",
%%%     filename        = "epsf.tex",
%%%     address         = "Tom Rokicki
%%%                        Box 2081
%%%                        Stanford, CA 94309
%%%                        USA",
%%%     telephone       = "+1 415 855 9989",
%%%     email           = "rokicki@cs.stanford.edu (Internet)",
%%%     codetable       = "ISO/ASCII",
%%%     keywords        = "PostScript, TeX",
%%%     supported       = "yes",
%%%     abstract        = "This file contains macros to support the inclusion
%%%                        of Encapsulated PostScript files in TeX documents.",
%%%     docstring       = "This file contains TeX macros to include an
%%%                        Encapsulated PostScript graphic.  It works
%%%                        by finding the bounding box comment,
%%%                        calculating the correct scale values, and
%%%                        inserting a vbox of the appropriate size at
%%%                        the current position in the TeX document.
%%%
%%%                        To use, simply say
%%%
%%%                        \input epsf % somewhere early on in your TeX file
%%%
%%%                        % then where you want to insert a vbox for a figure:
%%%                        \epsfbox{filename.ps}
%%%
%%%                        Alternatively, you can supply your own
%%%                        bounding box by
%%%
%%%                        \epsfbox[0 0 30 50]{filename.ps}
%%%
%%%                        This will not read in the file, and will
%%%                        instead use the bounding box you specify.
%%%
%%%                        The effect will be to typeset the figure as
%%%                        a TeX box, at the point of your \epsfbox
%%%                        command. By default, the graphic will have
%%%                        its `natural' width (namely the width of
%%%                        its bounding box, as described in
%%%                        filename.ps). The TeX box will have depth
%%%                        zero.
%%%
%%%                        You can enlarge or reduce the figure by
%%%                        saying
%%%
%%%                          \epsfxsize=<dimen> \epsfbox{filename.ps}
%%%                        or
%%%                          \epsfysize=<dimen> \epsfbox{filename.ps}
%%%
%%%                        instead. Then the width of the TeX box will
%%%                        be \epsfxsize and its height will be scaled
%%%                        proportionately (or the height will be
%%%                        \epsfysize and its width will be scaled
%%%                        proportionately).
%%%
%%%                        The width (and height) is restored to zero
%%%                        after each use, so \epsfxsize or \epsfysize
%%%                        must be specified before EACH use of
%%%                        \epsfbox.
%%%
%%%                        A more general facility for sizing is
%%%                        available by defining the \epsfsize macro.
%%%                        Normally you can redefine this macro to do
%%%                        almost anything.  The first parameter is
%%%                        the natural x size of the PostScript
%%%                        graphic, the second parameter is the
%%%                        natural y size of the PostScript graphic.
%%%                        It must return the xsize to use, or 0 if
%%%                        natural scaling is to be used.  Common uses
%%%                        include:
%%%
%%%                           \epsfxsize  % just leave the old value alone
%%%                           0pt         % use the natural sizes
%%%                           #1          % use the natural sizes
%%%                           \hsize      % scale to full width
%%%                           0.5#1       % scale to 50% of natural size
%%%                           \ifnum #1>\hsize\hsize\else#1\fi
%%%                                       % smaller of natural, hsize
%%%
%%%                        If you want TeX to report the size of the
%%%                        figure (as a message on your terminal when
%%%                        it processes each figure), say
%%%                        `\epsfverbosetrue'.
%%%
%%%                        If you only want to get the bounding box
%%%                        extents, without producing any output boxes
%%%                        or \special{}, then say
%%%                        \epsfgetbb{filename}.  The extents will be
%%%                        saved in the macros \epsfllx \epsflly
%%%                        \epsfurx \epsfury in PostScript units of
%%%                        big points.
%%%
%%%                        Revision history:
%%%
%%%                        ---------------------------------------------
%%%                        epsf.tex macro file:
%%%                        Originally written by Tomas Rokicki of
%%%                        Radical Eye Software, 29 Mar 1989.
%%%
%%%                        ---------------------------------------------
%%%                        Revised by Don Knuth, 3 Jan 1990.
%%%
%%%                        ---------------------------------------------
%%%                        Revised by Tomas Rokicki, 18 Jul 1990.
%%%                        Accept bounding boxes with no space after
%%%                        the colon.
%%%
%%%                        ---------------------------------------------
%%%                        Revised by Nelson H. F. Beebe
%%%                        <beebe@math.utah.edu>, 03 Dec 1991 [2.0].
%%%                        Add version number and date typeout.
%%%
%%%                        Use \immediate\write16 instead of \message
%%%                        to ensure output on new line.
%%%
%%%                        Handle nested EPS files.
%%%
%%%                        Handle %%BoundingBox: (atend) lines.
%%%
%%%                        Do not quit when blank lines are found.
%%%
%%%                        Add a few percents to remove generation of
%%%                        spurious blank space.
%%%
%%%                        Move \special output to
%%%                        \epsfspecial{filename} so that other macro
%%%                        packages can input this one, then change
%%%                        the definition of \epsfspecial to match
%%%                        another DVI driver.
%%%
%%%                        Move size computation to \epsfsetsize which
%%%                        can be called by the user; the verbose
%%%                        output of the bounding box and scaled width
%%%                        and height happens here.
%%%
%%%                        ---------------------------------------------
%%%                        Revised by Nelson H. F. Beebe
%%%                        <beebe@math.utah.edu>, 05 May 1992 [2.1].
%%%                        Wrap \leavevmode\hbox{} around \vbox{} with
%%%                        the \special so that \epsffile{} can be
%%%                        used inside \begin{center}...\end{center}
%%%
%%%                        ---------------------------------------------
%%%                        Revised by Nelson H. F. Beebe
%%%                        <beebe@math.utah.edu>, 09 Dec 1992 [2.2].
%%%                        Introduce \epsfshow{true,false} and
%%%                        \epsfframe{true,false} macros; the latter
%%%                        suppresses the insertion of the PostScript,
%%%                        and instead just creates an empty box,
%%%                        which may be handy for rapid prototyping.
%%%
%%%                        ---------------------------------------------
%%%                        Revised by Nelson H. F. Beebe
%%%                        <beebe@math.utah.edu>, 14 Dec 1992 [2.3].
%%%                        Add \epsfshowfilename{true,false}.  When
%%%                        true, and \epsfshowfalse is specified, the
%%%                        PostScript file name will be displayed
%%%                        centered in the figure box.
%%%
%%%                        ---------------------------------------------
%%%                        Revised by Nelson H. F. Beebe
%%%                        <beebe@math.utah.edu>, 20 June 1993 [2.4].
%%%                        Remove non-zero debug setting of \epsfframemargin,
%%%                        and change margin handling to preserve EPS image
%%%                        size and aspect ratio, so that the actual
%%%                        box is \epsfxsize+\epsfframemargin wide by
%%%                        \epsfysize+\epsfframemargin high.
%%%                        Reduce output of \epsfshowfilenametrue to
%%%                        just the bare file name.
%%%
%%%                        ---------------------------------------------
%%%                        Revised by Nelson H. F. Beebe
%%%                        <beebe@math.utah.edu>, 13 July 1993 [2.5].
%%%                        Add \epsfframethickness for control of
%%%                        \epsfframe frame lines.
%%%
%%%                        ---------------------------------------------
%%%                        Revised by Nelson H. F. Beebe
%%%                        <beebe@math.utah.edu>, 02 July 1996 [2.6]
%%%                        Add missing initialization \epsfatendfalse;
%%%                        the lack of this resulted in the wrong
%%%                        BoundingBox being picked up, mea culpa, sigh...
%%%                        ---------------------------------------------
%%%
%%%                        ---------------------------------------------
%%%                        Revised by Nelson H. F. Beebe
%%%                        <beebe@math.utah.edu>, 25 October 1996 [2.7]
%%%                        Update to match changes in from dvips 5-600
%%%                        distribution: new user-accessible macros:
%%%                        \epsfclipon, \epsfclipoff, \epsfdrafton,
%%%                        \epsfdraftoff, change \empty to \epsfempty.
%%%                        ---------------------------------------------
%%%                        
%%%                        Modified to avoid verbosity, give help.
%%%                        --kb@cs.umb.edu, for Texinfo.
%%%  }
%%% ====================================================================
%
\ifx\epsfannounce\undefined \def\epsfannounce{\immediate\write16}\fi
 \epsfannounce{This is `epsf.tex' v2.7k <10 July 1997>}%
%
\newread\epsffilein    % file to \read
\newif\ifepsfatend     % need to scan to LAST %%BoundingBox comment?
\newif\ifepsfbbfound   % success?
\newif\ifepsfdraft     % use draft mode?
\newif\ifepsffileok    % continue looking for the bounding box?
\newif\ifepsfframe     % frame the bounding box?
\newif\ifepsfshow      % show PostScript file, or just bounding box?
\epsfshowtrue          % default is to display PostScript file
\newif\ifepsfshowfilename % show the file name if \epsfshowfalse specified?
\newif\ifepsfverbose   % report what you're making?
\newdimen\epsfframemargin % margin between box and frame
\newdimen\epsfframethickness % thickness of frame rules
\newdimen\epsfrsize    % vertical size before scaling
\newdimen\epsftmp      % register for arithmetic manipulation
\newdimen\epsftsize    % horizontal size before scaling
\newdimen\epsfxsize    % horizontal size after scaling
\newdimen\epsfysize    % vertical size after scaling
\newdimen\pspoints     % conversion factor
%
\pspoints = 1bp        % Adobe points are `big'
\epsfxsize = 0pt       % default value, means `use natural size'
\epsfysize = 0pt       % ditto
\epsfframemargin = 0pt % default value: frame box flush around picture
\epsfframethickness = 0.4pt % TeX's default rule thickness
%
\def\epsfbox#1{\global\def\epsfllx{72}\global\def\epsflly{72}%
   \global\def\epsfurx{540}\global\def\epsfury{720}%
   \def\lbracket{[}\def\testit{#1}\ifx\testit\lbracket
   \let\next=\epsfgetlitbb\else\let\next=\epsfnormal\fi\next{#1}}%
%
% We use \epsfgetlitbb if the user specified an explicit bounding box,
% and \epsfnormal otherwise.  Because \epsfgetbb can be called
% separately to retrieve the bounding box, we move the verbose
% printing the bounding box extents and size on the terminal to
% \epsfstatus.  Therefore, when the user provided the bounding box,
% \epsfgetbb will not be called, so we must call \epsfsetsize and
% \epsfstatus ourselves.
%
\def\epsfgetlitbb#1#2 #3 #4 #5]#6{%
   \epsfgrab #2 #3 #4 #5 .\\%
   \epsfsetsize
   \epsfstatus{#6}%
   \epsfsetgraph{#6}%
}%
%
\def\epsfnormal#1{%
    \epsfgetbb{#1}%
    \epsfsetgraph{#1}%
}%
%
\newhelp\epsfnoopenhelp{The PostScript image file must be findable by
TeX, i.e., somewhere in the TEXINPUTS (or equivalent) path.}%
%
\def\epsfgetbb#1{%
%
%   The first thing we need to do is to open the
%   PostScript file, if possible.
%
    \openin\epsffilein=#1
    \ifeof\epsffilein
        \errhelp = \epsfnoopenhelp
        \errmessage{Could not open file #1, ignoring it}%
    \else                       %process the file
        {%                      %start a group to contain catcode changes
            % Make all special characters, except space, to be of type
            % `other' so we process the file in almost verbatim mode
            % (TeXbook, p. 344).
            \chardef\other=12
            \def\do##1{\catcode`##1=\other}%
            \dospecials
            \catcode`\ =10
            \epsffileoktrue         %true while we are looping
            \epsfatendfalse     %[02-Jul-1996]: add forgotten initialization
            \loop               %reading lines from the EPS file
                \read\epsffilein to \epsffileline
                \ifeof\epsffilein %then no more input
                \epsffileokfalse %so set completion flag
            \else                %otherwise process one line
                \expandafter\epsfaux\epsffileline:. \\%
            \fi
            \ifepsffileok
            \repeat
            \ifepsfbbfound
            \else
                \ifepsfverbose
                    \immediate\write16{No BoundingBox comment found in %
                                    file #1; using defaults}%
                \fi
            \fi
        }%                      %end catcode changes
        \closein\epsffilein
    \fi                         %end of file processing
    \epsfsetsize                %compute size parameters
    \epsfstatus{#1}%
}%
%
% Clipping control:
\def\epsfclipon{\def\epsfclipstring{ clip}}%
\def\epsfclipoff{\def\epsfclipstring{\ifepsfdraft\space clip\fi}}%
\epsfclipoff % default for dvips is OFF
%
% The special that is emitted by \epsfsetgraph comes from this macro.
% It is defined separately to allow easy customization by other
% packages that first \input epsf.tex, then redefine \epsfspecial.
% This macro is invoked in the lower-left corner of a box of the
% width and height determined from the arguments to \epsffile, or
% from the %%BoundingBox in the EPS file itself.
%
% This version is for dvips:
\def\epsfspecial#1{%
     \epsftmp=10\epsfxsize
     \divide\epsftmp\pspoints
     \ifnum\epsfrsize=0\relax
       \special{PSfile=\ifepsfdraft psdraft.ps\else#1\fi\space
                llx=\epsfllx\space
                lly=\epsflly\space
                urx=\epsfurx\space
                ury=\epsfury\space
                rwi=\number\epsftmp
                \epsfclipstring
               }%
     \else
       \epsfrsize=10\epsfysize
       \divide\epsfrsize\pspoints
       \special{PSfile=\ifepsfdraft psdraft.ps\else#1\fi\space
                llx=\epsfllx\space
                lly=\epsflly\space
                urx=\epsfurx\space
                ury=\epsfury\space
                rwi=\number\epsftmp\space
                rhi=\number\epsfrsize
                \epsfclipstring
               }%
     \fi
}%
%
% \epsfframe macro adapted from the TeXbook, exercise 21.3, p. 223, 331.
% but modified to set the box width to the natural width, rather
% than the line width, and to include space for margins and rules
\def\epsfframe#1%
{%
  \leavevmode                   % so we can put this inside
                                % a centered environment
  \setbox0 = \hbox{#1}%
  \dimen0 = \wd0                                % natural width of argument
  \advance \dimen0 by 2\epsfframemargin         % plus width of 2 margins
  \advance \dimen0 by 2\epsfframethickness      % plus width of 2 rule lines
  \vbox
  {%
    \hrule height \epsfframethickness depth 0pt
    \hbox to \dimen0
    {%
      \hss
      \vrule width \epsfframethickness
      \kern \epsfframemargin
      \vbox {\kern \epsfframemargin \box0 \kern \epsfframemargin }%
      \kern \epsfframemargin
      \vrule width \epsfframethickness
      \hss
    }% end hbox
    \hrule height 0pt depth \epsfframethickness
  }% end vbox
}%
%
\def\epsfsetgraph#1%
{%
   %
   % Make the vbox and stick in a \special that the DVI driver can
   % parse.  \vfil and \hfil are used to place the \special origin at
   % the lower-left corner of the vbox.  \epsfspecial can be redefined
   % to produce alternate \special syntaxes.
   %
   \leavevmode
   \hbox{% so we can put this in \begin{center}...\end{center}
     \ifepsfframe\expandafter\epsfframe\fi
     {\vbox to\epsfysize
     {%
        \ifepsfshow
            % output \special{} at lower-left corner of figure box
            \vfil
            \hbox to \epsfxsize{\epsfspecial{#1}\hfil}%
        \else
            \vfil
            \hbox to\epsfxsize{%
               \hss
               \ifepsfshowfilename
               {%
                  \epsfframemargin=3pt % local change of margin
                  \epsfframe{{\tt #1}}%
               }%
               \fi
               \hss
            }%
            \vfil
        \fi
     }%
   }}%
   %
   % Reset \epsfxsize and \epsfysize, as documented above.
   %
   \global\epsfxsize=0pt
   \global\epsfysize=0pt
}%
%
%   Now we have to calculate the scale and offset values to use.
%   First we compute the natural sizes.
%
\def\epsfsetsize
{%
   \epsfrsize=\epsfury\pspoints
   \advance\epsfrsize by-\epsflly\pspoints
   \epsftsize=\epsfurx\pspoints
   \advance\epsftsize by-\epsfllx\pspoints
%
%   If `epsfxsize' is 0, we default to the natural size of the picture.
%   Otherwise we scale the graph to be \epsfxsize wide.
%
   \epsfxsize=\epsfsize{\epsftsize}{\epsfrsize}%
   \ifnum \epsfxsize=0
      \ifnum \epsfysize=0
        \epsfxsize=\epsftsize
        \epsfysize=\epsfrsize
        \epsfrsize=0pt
%
%   We have a sticky problem here:  TeX doesn't do floating point arithmetic!
%   Our goal is to compute y = rx/t. The following loop does this reasonably
%   fast, with an error of at most about 16 sp (about 1/4000 pt).
%
      \else
        \epsftmp=\epsftsize \divide\epsftmp\epsfrsize
        \epsfxsize=\epsfysize \multiply\epsfxsize\epsftmp
        \multiply\epsftmp\epsfrsize \advance\epsftsize-\epsftmp
        \epsftmp=\epsfysize
        \loop \advance\epsftsize\epsftsize \divide\epsftmp 2
        \ifnum \epsftmp>0
           \ifnum \epsftsize<\epsfrsize
           \else
              \advance\epsftsize-\epsfrsize \advance\epsfxsize\epsftmp
           \fi
        \repeat
        \epsfrsize=0pt
      \fi
   \else
     \ifnum \epsfysize=0
       \epsftmp=\epsfrsize \divide\epsftmp\epsftsize
       \epsfysize=\epsfxsize \multiply\epsfysize\epsftmp
       \multiply\epsftmp\epsftsize \advance\epsfrsize-\epsftmp
       \epsftmp=\epsfxsize
       \loop \advance\epsfrsize\epsfrsize \divide\epsftmp 2
       \ifnum \epsftmp>0
          \ifnum \epsfrsize<\epsftsize
          \else
             \advance\epsfrsize-\epsftsize \advance\epsfysize\epsftmp
          \fi
       \repeat
       \epsfrsize=0pt
     \else
       \epsfrsize=\epsfysize
     \fi
   \fi
}%
%
% Issue some status messages if the user requested them
%
\def\epsfstatus#1{% arg = filename
   \ifepsfverbose
     \immediate\write16{#1: BoundingBox:
                  llx = \epsfllx\space lly = \epsflly\space
                  urx = \epsfurx\space ury = \epsfury\space}%
     \immediate\write16{#1: scaled width = \the\epsfxsize\space
                  scaled height = \the\epsfysize}%
   \fi
}%
%
%   We still need to define the tricky \epsfaux macro. This requires
%   a couple of magic constants for comparison purposes.
%
{\catcode`\%=12 \global\let\epsfpercent=%\global\def\epsfbblit{%BoundingBox}}%
\global\def\epsfatend{(atend)}%
%
%   So we're ready to check for `%BoundingBox:' and to grab the
%   values if they are found.
%
%   If we find a line
%
%   %%BoundingBox: (atend)
%
%   then we ignore it, but set a flag to force parsing all of the
%   file, so the last %%BoundingBox parsed will be the one used.  This
%   is necessary, because EPS files can themselves contain other EPS
%   files with their own %%BoundingBox comments.
%
%   If we find a line
%
%   %%BoundingBox: llx lly urx ury
%
%   then we save the 4 values in \epsfllx, \epsflly, \epsfurx, \epsfury.
%   Then, if we have not previously parsed an (atend), we flag completion
%   and can stop reading the file.  Otherwise, we must keep on reading
%   to end of file so that we find the values on the LAST %%BoundingBox.
\long\def\epsfaux#1#2:#3\\%
{%
   \def\testit{#2}%             % save second character up to just before colon
   \ifx#1\epsfpercent           % then first char is percent (quick test)
       \ifx\testit\epsfbblit    % then (slow test) we have %%BoundingBox
            \epsfgrab #3 . . . \\%
            \ifx\epsfllx\epsfatend % then ignore %%BoundingBox: (atend)
                \global\epsfatendtrue
            \else               % else found %%BoundingBox: llx lly urx ury
                \ifepsfatend    % then keep parsing ALL %%BoundingBox lines
                \else           % else stop after first one parsed
                    \epsffileokfalse
                \fi
                \global\epsfbbfoundtrue
            \fi
       \fi
   \fi
}%
%
%   Here we grab the values and stuff them in the appropriate definitions.
%
\def\epsfempty{}%
\def\epsfgrab #1 #2 #3 #4 #5\\{%
   \global\def\epsfllx{#1}\ifx\epsfllx\epsfempty
      \epsfgrab #2 #3 #4 #5 .\\\else
   \global\def\epsflly{#2}%
   \global\def\epsfurx{#3}\global\def\epsfury{#4}\fi
}%
%
%   We default the epsfsize macro.
%
\def\epsfsize#1#2{\epsfxsize}%
%
%   Finally, another definition for compatibility with older macros.
%
\let\epsffile=\epsfbox
\endinput
                            % to allow postscript inclusions
% On thor and CUS read top of file:
%     /opt/TeX/lib/texmf/tex/dvips/epsf.sty
% On CL machines read:
%     /usr/lib/tex/macros/dvips/epsf.tex



\raggedbottom                           % try to avoid widows and orphans
\sloppy
\clubpenalty1000%
\widowpenalty1000%

\addtolength{\oddsidemargin}{6mm}       % adjust margins
\addtolength{\evensidemargin}{-8mm}

\renewcommand{\baselinestretch}{1.1}    % adjust line spacing to make
                                        % more readable

\begin{document}

\bibliographystyle{plain}


%%%%%%%%%%%%%%%%%%%%%%%%%%%%%%%%%%%%%%%%%%%%%%%%%%%%%%%%%%%%%%%%%%%%%%%%
% Title


\pagestyle{empty}

\hfill{\LARGE \bf Philip Yeeles}

\vspace*{60mm}
\begin{center}
\Huge
{\bf Nico: An Environment for Mathematical Expression in Schools} \\
\vspace*{5mm}
Computer Science Tripos \\
\vspace*{5mm}
Selwyn College \\
\vspace*{5mm}
\today  % today's date
\end{center}

\cleardoublepage

%%%%%%%%%%%%%%%%%%%%%%%%%%%%%%%%%%%%%%%%%%%%%%%%%%%%%%%%%%%%%%%%%%%%%%%%%%%%%%
% Proforma, table of contents and list of figures

\setcounter{page}{1}
\pagenumbering{roman}
\pagestyle{plain}

\chapter*{Proforma}

{\large
\begin{tabular}{ll}
Name:               & \bf Philip Yeeles                                               \\
College:            & \bf Selwyn College                                              \\
Project Title:      & \bf Nico: An Environment for Mathematical\\
                    & \bf Expression in Schools \\
Examination:        & \bf Computer Science Tripos, May 2012                           \\
Word Count:         & \bf TBC\footnotemark[1]
(well less than the 12000 limit) \\
Project Originator: & P.~M.~Yeeles (\verb¬pmy22¬)                                    \\
Supervisors:        & Dr S.~J.~Aaron (\verb¬sja55¬), A.~G.~Stead (\verb¬ags46¬)     \\
\end{tabular}
}
\footnotetext[1]{This word count was computed
by {\tt detex diss.tex | tr -cd '0-9A-Za-z $\tt\backslash$n' | wc -w}
}
\stepcounter{footnote}


\section*{Original Aims of the Project}

To write a demonstration dissertation\footnote{A normal footnote without the
complication of being in a table.} using \LaTeX\ to save
student's time when writing their own dissertations. The dissertation
should illustrate how to use the more common \LaTeX\ constructs. It
should include pictures and diagrams to show how these can be
incorporated into the dissertation.  It should contain the entire
\LaTeX\ source of the dissertation and the Makefile.  It should
explain how to construct an MSDOS disk of the dissertation in
Postscript format that can be used by the book shop for printing, and,
finally, it should have the prescribed layout and format of a diploma
dissertation.


\section*{Work Completed}

All that has been completed appears in this dissertation.

\section*{Special Difficulties}

Learning how to incorporate encapulated postscript into a \LaTeX\
document on both CUS and Thor.

\newpage
\section*{Declaration}

I, Philip Michael Yeeles of Selwyn College, being a candidate for Part
II of the Computer Science Tripos, hereby declare that this dissertation
and the work described in it are my own work, unaided except as may be
specified below, and that the dissertation does not contain material
that has already been used to any substantial extent for a comparable
purpose.

\bigskip
\leftline{Signed}

\medskip
\leftline{Date \today}

\cleardoublepage

\tableofcontents

\listoffigures

\newpage
\section*{Acknowledgements}

This document owes much to an earlier version written by Simon Moore
\cite{Moore95}.  His help, encouragement and advice was greatly
appreciated.

%%%%%%%%%%%%%%%%%%%%%%%%%%%%%%%%%%%%%%%%%%%%%%%%%%%%%%%%%%%%%%%%%%%%%%%
% now for the chapters

\cleardoublepage        % just to make sure before the page numbering
                        % is changed

\setcounter{page}{1}
\pagenumbering{arabic}
\pagestyle{headings}

\chapter{Introduction}

\section{Overview of the files}

This document consists of the following files:

\begin{itemize}
\item {\tt Makefile} --- The Makefile for the dissertation and Project Proposal
\item {\tt diss.tex} --- The dissertation
\item {\tt propbody.tex} --- Appendix~C  -- the project proposal
\item {\tt proposal.tex}  --- A \LaTeX\ main file for the proposal
\item{\tt figs} -- A directory containing diagrams and pictures
\item{\tt refs.bib} --- The bibliography database
\end{itemize}

\section{Building the document}

This document was produced using \LaTeXe which is based upon
\LaTeX\cite{Lamport86}.  To build the document you first need to
generate {\tt diss.aux} which, amongst other things, contains the
references used.  This if done by executing the command:

{\tt latex diss}

\noindent
Then the bibliography can be generated from {\tt refs.bib} using:

{\tt bibtex diss}

\noindent
Finally, to ensure all the page numbering is correct run {\tt latex}
on {\tt diss.tex} until the {\tt .aux} files do not change.  This
usually takes 2 more runs.

\subsection{The makefile}

To simplify the calls to {\tt latex} and {\tt bibtex},
a makefile has been provided, see Appendix~\ref{makefile}.
It provides the following facilities:

\begin{itemize}

\item{\tt make} \\
 Display help information.

\item{\tt make prop} \\
 Run {\tt latex proposal; xdvi proposal.dvi}.

\item{\tt make diss.ps} \\
 Make the file {\tt diss.ps}.

\item{\tt make gv} \\
 View the dissertation using ghostview after performing
{\tt make diss.ps}, if necessary.

\item{\tt make gs} \\
 View the dissertation using ghostscript after performing
{\tt make diss.ps}, if necessary.

\item{\tt make count} \\
Display an estimate of the word count.

\item{\tt make all} \\
Construct {\tt proposal.dvi} and {\tt diss.ps}.

\item{\tt make pub} \\ Make a {\tt .tar} version of the {\tt demodiss}
directory and place it in my {\tt public\_html} directory.

\item{\tt make clean} \\ Delete all files except the source files of
the dissertation. All these deleted files can be reconstructed by
typing {\tt make all}.

\item{\tt make pr} \\
Print the dissertation on your default printer.

\end{itemize}


\section{Counting words}

An approximate word count of the body of the dissertation may be
obtained using:

{\tt wc diss.tex}

\noindent
Alternatively, try something like:

\verb/detex diss.tex | tr -cd '0-9A-Z a-z\n' | wc -w/




\cleardoublepage



\chapter{Preparation}

This chapter is empty!


\cleardoublepage
\chapter{Implementation}

\section{Verbatim text}

Verbatim text can be included using \verb|\begin{verbatim}| and
\verb|\end{verbatim}|. I normally use a slightly smaller font and
often squeeze the lines a little closer together, as in:

{\renewcommand{\baselinestretch}{0.8}\small\begin{verbatim}
GET "libhdr"

GLOBAL { count:200; all  }

LET try(ld, row, rd) BE TEST row=all
                        THEN count := count + 1
                        ELSE { LET poss = all & ~(ld | row | rd)
                               UNTIL poss=0 DO
                               { LET p = poss & -poss
                                 poss := poss - p
                                 try(ld+p << 1, row+p, rd+p >> 1)
                               }
                             }
LET start() = VALOF
{ all := 1
  FOR i = 1 TO 12 DO
  { count := 0
    try(0, 0, 0)
    writef("Number of solutions to %i2-queens is %i5*n", i, count)
    all := 2*all + 1
  }
  RESULTIS 0
}
\end{verbatim}
}

\section{Tables}

\begin{samepage}
Here is a simple example\footnote{A footnote} of a table.

\begin{center}
\begin{tabular}{l|c|r}
Left      & Centred & Right \\
Justified &         & Justified \\[3mm]
%\hline\\%[-2mm]
First     & A       & XXX \\
Second    & AA      & XX  \\
Last      & AAA     & X   \\
\end{tabular}
\end{center}

\noindent
There is another example table in the proforma.
\end{samepage}

\section{Simple diagrams}

Simple diagrams can be written directly in \LaTeX.  For example, see
figure~\ref{latexpic1} on page~\pageref{latexpic1} and see
figure~\ref{latexpic2} on page~\pageref{latexpic2}.

\begin{figure}
\setlength{\unitlength}{1mm}
\begin{center}
\begin{picture}(125,100)
\put(0,80){\framebox(50,10){AAA}}
\put(0,60){\framebox(50,10){BBB}}
\put(0,40){\framebox(50,10){CCC}}
\put(0,20){\framebox(50,10){DDD}}
\put(0,00){\framebox(50,10){EEE}}

\put(75,80){\framebox(50,10){XXX}}
\put(75,60){\framebox(50,10){YYY}}
\put(75,40){\framebox(50,10){ZZZ}}

\put(25,80){\vector(0,-1){10}}
\put(25,60){\vector(0,-1){10}}
\put(25,50){\vector(0,1){10}}
\put(25,40){\vector(0,-1){10}}
\put(25,20){\vector(0,-1){10}}

\put(100,80){\vector(0,-1){10}}
\put(100,70){\vector(0,1){10}}
\put(100,60){\vector(0,-1){10}}
\put(100,50){\vector(0,1){10}}

\put(50,65){\vector(1,0){25}}
\put(75,65){\vector(-1,0){25}}
\end{picture}
\end{center}
\caption{\label{latexpic1}A picture composed of boxes and vectors.}
\end{figure}

\begin{figure}
\setlength{\unitlength}{1mm}
\begin{center}

\begin{picture}(100,70)
\put(47,65){\circle{10}}
\put(45,64){abc}

\put(37,45){\circle{10}}
\put(37,51){\line(1,1){7}}
\put(35,44){def}

\put(57,25){\circle{10}}
\put(57,31){\line(-1,3){9}}
\put(57,31){\line(-3,2){15}}
\put(55,24){ghi}

\put(32,0){\framebox(10,10){A}}
\put(52,0){\framebox(10,10){B}}
\put(37,12){\line(0,1){26}}
\put(37,12){\line(2,1){15}}
\put(57,12){\line(0,2){6}}
\end{picture}

\end{center}
\caption{\label{latexpic2}A diagram composed of circles, lines and boxes.}
\end{figure}



\section{Adding more complicated graphics}

The use of \LaTeX\ format can be tedious and it is often better to use
encapsulated postscript to represent complicated graphics.
Figure~\ref{epsfig} and ~\ref{xfig} on page \pageref{xfig} are
examples. The second figure was drawn using {\tt xfig} and exported in
{\tt.eps} format. This is my recommended way of drawing all diagrams.


\begin{figure}[tbh]
\centerline{\epsfbox{figs/cuarms.eps}}
\caption{\label{epsfig}Example figure using encapsulated postscript}
\end{figure}

\begin{figure}[tbh]
\vspace{4in}
\caption{\label{pastedfig}Example figure where a picture can be pasted in}
\end{figure}


\begin{figure}[tbh]
\centerline{\epsfbox{figs/diagram.eps}}
\caption{\label{xfig}Example diagram drawn using {\tt xfig}}
\end{figure}




\cleardoublepage
\chapter{Evaluation}

\section{Printing and binding}

If you have access to a laser printer that can print on two sides, you
can use it to print two copies of your dissertation and then get them
bound by the Computer Laboratory Bookshop. Otherwise, print your
dissertation single sided and get the Bookshop to copy and bind it double
sided.


Better printing quality can sometimes be obtained by giving the
Bookshop an MSDOS 1.44~Mbyte 3.5" floppy disc containing the
Postscript form of your dissertation. If the file is too large a
compressed version with {\tt zip} but not {\tt gnuzip} nor {\tt
compress} is acceptable. However they prefer the uncompressed form if
possible. From my experience I do not recommend this method.

\subsection{Things to note}

\begin{itemize}
\item Ensure that there are the correct number of blank pages inserted
so that each double sided page has a front and a back.  So, for
example, the title page must be followed by an absolutely blank page
(not even a page number).

\item Submitted postscript introduces more potential problems.
Therefore you must either allow two iterations of the binding process
(once in a digital form, falling back to a second, paper, submission if
necessary) or submit both paper and electronic versions.

\item There may be unexpected problems with fonts.

\end{itemize}

\section{Further information}

See the Computer Lab's world wide web pages at URL:

{\tt http://www.cl.cam.ac.uk/TeXdoc/TeXdocs.html}


\cleardoublepage
\chapter{Conclusion}

I hope that this rough guide to writing a dissertation is \LaTeX\ has
been helpful and saved you time.




\cleardoublepage

%%%%%%%%%%%%%%%%%%%%%%%%%%%%%%%%%%%%%%%%%%%%%%%%%%%%%%%%%%%%%%%%%%%%%
% the bibliography

\addcontentsline{toc}{chapter}{Bibliography}
\bibliography{refs}
\cleardoublepage

%%%%%%%%%%%%%%%%%%%%%%%%%%%%%%%%%%%%%%%%%%%%%%%%%%%%%%%%%%%%%%%%%%%%%
% the appendices
\appendix

\chapter{Latex source}

\section{diss.tex}
{\scriptsize\verbatiminput{diss.tex}}

\section{proposal.tex}
{\scriptsize\verbatiminput{proposal.tex}}

\section{propbody.tex}
{\scriptsize\verbatiminput{propbody.tex}}



\cleardoublepage

\chapter{Makefile}

\section{\label{makefile}Makefile}
{\scriptsize\verbatiminput{makefile.txt}}

\section{refs.bib}
{\scriptsize\verbatiminput{refs.bib}}


\cleardoublepage

\chapter{Project Proposal}


% Draft #1 (final?)

\vfil

\centerline{\Large Diploma in Computer Science Project Proposal}
\vspace{0.4in}
\centerline{\Large How to write a dissertation in \LaTeX\ }
\vspace{0.4in}
\centerline{\large M. Richards, St John's College}
\vspace{0.3in}
\centerline{\large Originator: Dr M. Richards}
\vspace{0.3in}
\centerline{\large 21 November 2000}

\vfil

\subsection*{Special Resources Required}
File space on Thor -- 25Mbytes\\
Account on the DEC Workstations -- 15Mbytes\\
An account on Ouse\\
The use of my own IBM PC (1000GHz Pentium, 200Mb RAM and 40Gb Disk).
\vspace{0.2in}

\noindent
{\bf Project Supervisor:} Dr M. Richards
\vspace{0.2in}

\noindent
{\bf Director of Studies:} Dr M. Richards
\vspace{0.2in}
\noindent
 
\noindent
{\bf Project Overseers:} Dr~F.~H.~King  \& Dr~S.~W.~Moore

\vfil
\pagebreak

% Main document

\section*{Introduction}

Many students write their CST and Diploma dissertations in \LaTeX\ and
spend a fair amount of time learning just how to do that. The purpos of 
this project is to write a demonsatration dissertation that explains in
detail how it done and how the result can be given to the Bookshop
on an MSDOS floppy disk for printing and binding.

\section*{Work that has to be done}

The project breaks down into the following main sections:-

\begin{enumerate}

\item The construction of a skeleton dissertation with the required 
structure. This involves writing the Makefile and makeing dummy files
for the title page, the proforma, chapters 1 to 5, the appendices and
the proposal.

\item Filling in the details required in the cover page and proforma.

\item Writing the contents of chapters 1 to 5, including examples
of common \LaTeX\ constructs.

\item Adding a example of how to use floating figures and encapsulated
postscript diagrams.

\end{enumerate}

\section*{Difficulties to Overcome}

The following main learning tasks will have to be undertaken before 
the project can be started:

\begin{itemize}

\item To learn \LaTeX\ and its use on Thor.

\item To discover how to incorporate encapsulated postscript into
a \LaTeX\ document, and to find a suitable drawing package on Thor
to recommend.

\item To discover what format the Bookshop would like for the finished
dissertation, and how to deal with postscript files that are too
large to fit on a single floppy disk.

\end{itemize}



\section*{Starting Point}

I have a reasonable working knowledge of \LaTeX\ and have convenient
access to Thor using an IBM PC in my office. Writing MSDOS disks is no 
problem.

\section*{Resources}

This project requires little file space so 25Mbytes of disk space on Thor
should be sufficient. I plan to use my own IBM PC to write floppy disks, 
but could use the PWF PCs if my own machine breaks down. 

Backup will be on floppy disks.

\section*{Work Plan}

Planned starting date is 01/12/2000.

\subsection*{Michaelmas Term} 

By the end of this term I intend to have completed the learning tasks 
outlined in the relevant section.


\subsection*{Lent Term}

By the division of term the overall structure of the dissertation
will have been written and tested.

By the end of term, example figures using encapsulated postscript
will have been included.
 

\subsection*{Easter Term}

On completion of the exams I will incorporate final details into 
the dissertation including a bibliography using bibtex and a table of contents.
The estimated completion date being 25/07/2001 to allow 
plenty of time should any unforeseen problems arise.



\end{document}
