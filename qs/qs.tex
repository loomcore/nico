\documentclass[14pt]{article}

\usepackage[cm]{fullpage}
\usepackage{fancyhdr}
\usepackage{setspace}

\usepackage[xetex]{graphicx}
\usepackage{fontspec,xunicode}
\defaultfontfeatures{Mapping=tex-text,Scale=MatchLowercase, Numbers=OldStyle}
\setmainfont[Scale=1]{Linux Libertine O}
\setsansfont{Linux Biolinum O}
\setmonofont{Monaco}

\usepackage{wrapfig}

\pagestyle{fancy}
\lhead{Phil Yeeles}
\rhead{\today}

\setlength{\headsep}{1.25cm}
\setlength{\parindent}{0cm}
\setlength{\parskip}{0.25cm}

\newcommand{\likert}{
\begin{center}
% likert scale
\setlength{\unitlength}{5mm}
\begin{picture}(18,2)
\put(0,1.5){\circle{1}}
\put(0.5,1.5){\line(1,0){4}}
\put(5,1.5){\circle{1}}
\put(5.5,1.5){\line(1,0){4}}
\put(10,1.5){\circle{1}}
\put(10.5,1.5){\line(1,0){4}}
\put(15,1.5){\circle{1}}
\put(15.5,1.5){\line(1,0){4}}
\put(20,1.5){\circle{1}}
\put(-2.25,0){{\small Strongly disagree}}
\put(4,0){{\small Disagree}}
\put(9,0){{\small Neutral}}
\put(14.25,0){{\small Agree}}
\put(18,0){{\small Strongly agree}}
\end{picture}
\end{center}
}

\newcommand{\percentbox}{
% box to write a percentage answer in
\setlength{\unitlength}{5mm}
\begin{picture}(1,1)
\put(0,0){\line(1,0){1}}
\put(0,0){\line(0,1){1}}
\put(0,1){\line(1,0){1}}
\put(1,0){\line(0,1){1}}
\end{picture} \%}

\newcommand{\answerbox}{
\fbox{
\begin{minipage}{16cm}
\hfill\vspace{2cm}
\end{minipage}
}\\
}

\begin{document}

\begin{center}

{\LARGE {\bf \emph{Nico}: An Environment for Mathematical Expression in Schools}}

\end{center}

Thank you for agreeing to participate in the user study for my Part II Project.
\emph{Nico} is a piece of educational software designed to aid learners in the
visualisation of mathematical problems, by separating out the constituent parts
of a calculation into distinct visual units on-screen.

The purpose of this study is to ascertain how well \emph{Nico} achieves its goal
of providing a clear, accessible, interactive means of calculation, and to gather
feedback on how the application could be improved.  The study also aims to compare
\emph{Nico} to traditional mathematical methods.

Please review and sign the attached Statement of Informed Consent (Section 3)
and please feel free to ask any questions you may have about it.

Before we begin, please answer the following questions.

\begin{itemize}

\item Have you used \emph{Nico} before?

\begin{center}
Yes\hspace{1cm}No\\
\end{center}

\item How well do you agree with the following statement (cross as appropriate)?\\
\emph{I am confident in my ability to calculate answers to simple mathematical problems.}

\likert

\end{itemize}

Thank you.  Let us begin.

\section{Study}

You will now be issued a group number.  If you are in Group 1, please proceed to
Section 1.1.  If you are in Group 2, please proceed to Section 1.2.  If you would
like to stop at any point, please don't hesitate to let me know.

\subsection{\emph{Nico}}

In this section, you will use \emph{Nico} to solve some simple mathematical
problems.  The purpose of this section is to evaluate how well \emph{Nico}
performs in comparison with manual calculation.  We will be keeping a record of
the time taken to complete each problem, but please do not let this make you feel
rushed.  Work at a pace that is normal and comfortable for you.

Before we begin the tasks, please watch the instructional video
\verb¬qs/tut.avi¬ for a briefing on how to use \emph{Nico} and an explanation
of its controls.

Now that you have done this, please spend 5 minutes experimenting with
\emph{Nico}.  Open the application and load the file \verb¬qs/blank.nqs¬ using
the file chooser.  As you explore, please tell me your thoughts about the
application.

Now, let us move on to the problems.  Please use the {\sffamily {\bf File}} menu
at the top of the screen to load the file \verb¬qs/user-study.nqs¬.  You will
be presented with a series of problems to solve using \emph{Nico}; please solve
them.

Thank you very much.

\begin{itemize}
\item If you are in Group 1, please continue to Section 1.2
\item If you are in Group 2, please continue to Section 2
\end{itemize}

\subsection{Manual Calculation}

In this section, you will solve some simple mathematical problems using pen
and paper.  The purpose of this section is as a control, to compare to your
results using \emph{Nico}.  Once again, we will be keeping a record of the time
you take to complete each question, but please do not let this make you feel
rushed.  Work at a pace that is normal and comfortable for you, and don't forget
to show your working.

Let us begin.

% possibly present on a separate sheet so they can't look ahead and cheat?

\begin{enumerate}

\item \emph{2+3}\\

\fbox{
\begin{minipage}{16cm}
\hfill\vspace{3cm}
\end{minipage}
}\\

\item \emph{9÷3}\\

\fbox{
\begin{minipage}{16cm}
\hfill\vspace{3cm}
\end{minipage}
}\\

\item \emph{1+2+3+4+5}\\

\fbox{
\begin{minipage}{16cm}
\hfill\vspace{3cm}
\end{minipage}
}\\

\item \emph{(2×4)+(3-5)}\\

\fbox{
\begin{minipage}{16cm}
\hfill\vspace{3cm}
\end{minipage}
}\\

\pagebreak

\item \emph{((3×4)÷(3+3))×8}\\

\fbox{
\begin{minipage}{16cm}
\hfill\vspace{3cm}
\end{minipage}
}\\

\item \emph{((1+2+3+4+5)+(2×4×6×8×10))×1×2×(1+2+3+4+5)}\\

\fbox{
\begin{minipage}{16cm}
\hfill\vspace{3cm}
\end{minipage}
}\\

\item \emph{12+14}\\

\fbox{
\begin{minipage}{16cm}
\hfill\vspace{3cm}
\end{minipage}
}\\

\item \emph{247×35}\\

\fbox{
\begin{minipage}{16cm}
\hfill\vspace{3cm}
\end{minipage}
}\\

\item \emph{120÷((2×10)+5+5)}\\

\fbox{
\begin{minipage}{16cm}
\hfill\vspace{3cm}
\end{minipage}
}\\

\pagebreak

\item \emph{((2+5)×(6÷2)×(9-8))+((3+4)-(5×6))+120}\\

\fbox{
\begin{minipage}{16cm}
\hfill\vspace{3cm}
\end{minipage}
}\\

\end{enumerate}

Thank you very much.

\begin{itemize}
\item If you are in Group 1, please continue to Section 2
\item If you are in Group 2, please continue to Section 1.1
\end{itemize}

\section{Questions}

\subsection{Notation}

When using the system, what proportion of your time (as a rough percentage) do you spend:

\begin{enumerate}
\item Searching for information within the notation \percentbox
\item Translating substantial amounts of information from some other source into the system \percentbox
\item Adding small bits of information to a description that you have previously created \percentbox
\item Reorganising and restructuring descriptions that you have previously created \percentbox
\item Playing around with new ideas in the notation, without being sure what will result \percentbox
\end{enumerate}

\subsection{Cognitive Dimensions}

\subsubsection{Visibility and Juxtaposability}

\begin{enumerate}
\item How easy is it to see or find the various parts of the notation while it is being created or changed? Why?\\
\answerbox
\item What kind of things are more difficult to see or find?\\
\answerbox
\item If you need to compare or combine different parts, can you see them at the same time? If not, why not?\\
\answerbox
\end{enumerate}

\subsubsection{Viscosity}

\begin{enumerate}
\item When you need to make changes to previous work, how easy is it to make the change? Why?\\
\answerbox
\item Are there particular changes that are more difficult or especially difficult to make? Which ones?\\
\answerbox
\end{enumerate}

%\subsubsection{Diffuseness}
%
%\begin{enumerate}
%\item Does the notation let you say what you want reasonably briefly, or is it long-winded? Why?\\
%\answerbox
%\item What sorts of things take more space to describe?\\
%\answerbox
%\end{enumerate}

%\subsubsection{Hard Mental Operations}
%
%\begin{enumerate}
%\item What kind of things require the most mental effort with this notation?\\
%\answerbox
%\item Do some things seem especially complex or difficult to work out in your head (e.g. when combining several things)?  What are they?\\
%\answerbox
%\end{enumerate}

\subsubsection{Error Proneness}

\begin{enumerate}
\item Do some kinds of mistake seem particularly common or easy to make? Which ones?\\
\answerbox
\item Do you often find yourself making small slips that irritate you or make you feel stupid? What are some examples?\\
\answerbox
\end{enumerate}

\subsubsection{Closeness of Mapping}

\begin{enumerate}
\item How closely related is the notation to the result that you are describing? Why?\\ % (Note that in a sub-device, the result may be part of another notation, rather than the end product).
\answerbox

\pagebreak

\item Which parts seem to be a particularly strange way of doing or describing something?\\
\answerbox
\end{enumerate}

\subsubsection{Role Expressiveness}

\begin{enumerate}
\item When reading the notation, is it easy to tell what each part is for in the overall scheme? Why?\\
\answerbox
\item Are there some parts that are particularly difficult to interpret? Which ones?\\
\answerbox
\item Are there parts that you really don’t know what they mean, but you put them in just because it’s always been that way? What are they?\\
\answerbox
\end{enumerate}

\subsubsection{Hidden Dependencies}

\begin{enumerate}
\item If the structure of the product means some parts are closely related to other parts, and changes to one may affect the other, are those dependencies visible? What kind of dependencies are hidden?\\
\answerbox
\item In what ways can it get worse when you are creating a particularly large description?\\
\answerbox

\pagebreak

\item Do these dependencies stay the same, or are there some actions that cause them to get frozen? If so, what are they?\\
\answerbox
\end{enumerate}

\subsubsection{Progressive Evaluation}

\begin{enumerate}
\item How easy is it to stop in the middle of creating some notation, and check your work so far? Can you do this any time you like? If not, why not?\\
\answerbox
\item Can you find out how much progress you have made, or check what stage in your work you are up to? If not, why not?\\
\answerbox
\item Can you try out partially-completed versions of the product? If not, why not?\\
\answerbox
\end{enumerate}

\subsubsection{Provisionality}

\begin{enumerate}
\item Is it possible to sketch things out when you are playing around with ideas, or when you aren’t sure which way to proceed? What features of the notation help you to do this?\\
\answerbox
\item What sort of things can you do when you don’t want to be too precise about the exact result you are trying to get?\\
\answerbox
\end{enumerate}

%\subsubsection{Premature Commitment}
%
%\begin{enumerate}
%\item When you are working with the notation, can you go about the job in any order you like, or does the system force you to think ahead and make certain decisions first?\\
%\answerbox
%\item If so, what decisions do you need to make in advance? What sort of problems can this cause in your work?\\
%\answerbox
%\end{enumerate}

%\subsubsection{Consistency}
%
%\begin{enumerate}
%\item Where there are different parts of the notation that mean similar things, is the similarity clear from the way they appear? Please give examples.\\
%\answerbox
%\item Are there places where some things ought to be similar, but the notation makes them different? What are they?\\
%\answerbox
%\end{enumerate}

\subsubsection{Secondary Notation}

\begin{enumerate}
\item Is it possible to make notes to yourself, or express information that is not really recognised as part of the notation?\\
\answerbox
\item If it was printed on a piece of paper that you could annotate or scribble on, what would you write or draw?\\
\answerbox
\item Do you ever add extra marks (or colours or format choices) to clarify, emphasise or repeat what is there already?\\ % [If yes: does this constitute a helper device? If so, please fill in one of the section 5 sheets describing it]
\answerbox
\end{enumerate}

%\subsubsection{Abstraction Management}
%
%\begin{enumerate}
%\item Does the system give you any way of defining new facilities or terms within the notation, so that you can extend it to describe new things or to express your ideas more clearly or succinctly? What are they?\\
%\answerbox
%\item Does the system insist that you start by defining new terms before you can do anything else? What sort of things?\\
%\answerbox
%\item If you wrote here, you have a redefinition device: please fill in one of the section 5 sheets describing it.\\
%\answerbox
%\end{enumerate}

\subsection{Feedback}

\begin{enumerate}
\item Do you find yourself using this notation in ways that are unusual, or ways that the designer might not have intended? If so, what are some examples?\\
\answerbox
\item After completing this questionnaire, can you think of obvious ways that the design of the system could be improved? What are they? Could it be improved specifically for your own requirements?\\
\answerbox
\end{enumerate}

\section{Statement of Informed Consent}

Thank you very much for your help with my project!  Your responses will remain
confidential, although they are liable to appear in an anonymised form
in the final report, of which a copy will be retained by the University of
Cambridge Computer Laboratory Library
(\verb¬http://www.cl.cam.ac.uk/library/¬).  The final report will also be
available via my repository at GitHub
(\verb¬http://github.com/loomcore/nico¬).

\begin{center}

\fbox{
\parbox{18cm}{
\begin{center}
{\large {\bf Statement of Informed Consent}}\\\vspace{0.5cm}
\end{center}

I state that I am over 18 years of age and wish to participate in a program of
research being conducted by Philip Yeeles at the University of Cambridge. I
acknowledge that this study has been approved by the University of Cambridge
Computer Laboratory Ethics Committee.\\

The purpose of this research is to assess the usability of a prototype user
interface for representing mathematical calculations in a graphical manner.\\

The procedures involve the monitored use of the interface.  I will be asked to
perform specific tasks using the tools.  I will also be asked open-ended
questions about the tools and my experience using them.\\

All information collected in the study is confidential, and my name will not
be identified at any time.  I understand that I am free to ask questions or to
withdraw from participation at any time without penalty.\\

I acknowledge that my (anonymised) responses may be published in the final
report, and that this report will be made publicly available from the University
of Cambridge Computer Laboratory Library and from GitHub.\\

\begin{spacing}{2}
{\bf Signed:}
\end{spacing}
\begin{spacing}{1.5}
{\bf Name:}\\
{\bf Date:}\\
\end{spacing}
}
}

\end{center}

\end{document}
